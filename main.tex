\documentclass{article}
\usepackage{graphicx} % Required for inserting images
\usepackage{amssymb, amsmath, amsthm}
\usepackage{mathtools}
\usepackage{url}
\usepackage{graphicx} % Required for inserting images
\newtheorem{remark}{Remark}
\newtheorem{definition}{Definition}
\newtheorem{theorem}{Theorem}
\newtheorem{lemma}{Lemma}
\newtheorem{example}{Example}
\newtheorem{conjecture}{Conjecture}
\title{Honor Thesis on Hamiltonian Circuit in Cayley Graphs}
\author{Jiaming Zhang}
\date{April 2025}

\begin{document}
\maketitle
\section{Thesis Overview}
This thesis addresses a long-standing open problem in algebraic graph theory: 
\begin{conjecture}[Lov\'asz-type Conjecture for Cayley Graphs]
Every finite connected Cayley graph is Hamiltonian.
\end{conjecture}

While partial progress has been made over the decades, a complete resolution remains elusive. In exploring the foundational literature, I carefully analyzed a 1983 paper by Dragan Marušić, which presented a construction intended to prove the Hamiltonicity of Cayley graphs for certain group classes.~\cite{marusic1983} Upon close examination, I identified a critical gap in the proof—specifically, a lack of rigorous justification for the completeness of the proposed algorithm. Motivated by this, my thesis establishes a corrected and extended version of Marušić’s method, providing a formal proof for the algorithm’s ability to generate Hamiltonian sequences (Lemma 3.2 below). The core contribution is a recursive construction framework that, when combined with Marušić’s original approach, produces dynamic Hamiltonian paths across group layers. This work not only fills a crucial logical gap in a well-cited result, but also contributes a validated method that may aid further exploration of Hamiltonian properties in group-theoretic graph structures.~\cite{stelow2017}~\cite{bajo2021}

\section{Preliminaries}
In this section, we keep the original notation and definition from Marušić's paper.

\begin{definition} 
If $\Gamma$ is a graph, then $V(\Gamma)$ and $E(\Gamma)$ will denote the set of vertices and the set of edges of $\Gamma$, and A graph $\Gamma$ is said to be \textit{Hamiltonian} if it has a \textit{Hamiltonian} circuit, that is, a circuit of length $|V(F)|$.
\end{definition}

\begin{definition} 
Let $G$ be a group, with identity element $e$. If $g \in G$, then $|g|$ will denote the order of $g$. If $M \subseteq G$, then we define the following
\paragraph{(a)} 
$M^{-1} = \{x^{-1} : x \in M\}$, the set of inverses of elements in $M$.
\paragraph{(b)}
$M_0 = M - \{\text{e}\}$, the set of elements in $M$ excluding the identity element.
\paragraph{(c)} 
$M^* = M_0 \cup M_0^{-1}$, the union of $M_0$ and the set of inverses of $M_0$.
\paragraph{(d)} 
$M$ = The subgroup of $G$ generated by $M$.
\end{definition}
\begin{example}
Let $G=\mathbb{Z}_{12}$, $M = \{0, 2, 3, 6\}$. Then, $M^{-1} = \{0, 6, 9, 10\}$, $M_0 = \{2, 3, 6\}$, $M^* = M_0 \cup M_0^{-1} = \{2, 3, 6\}\cup\{10, 9, 6\} = \{2, 3, 10, 9, 6\}$, and $M = \langle M\rangle = \mathbb{Z}_{12}$.
\end{example}

\begin{definition}[Sequence] 
A \textit{sequence} $S = [s_1, s_2, \dots, s_r]$ on $G$ is a sequence all of whose terms are elements of $G$. By $\Box$ we shall denote the empty sequence on $G$, that is, the sequence with no terms. All other sequences on $G$ will be called non-empty. For a sequence $P$, denoted by $r_P$ its length.
\end{definition}

\begin{definition}[Partial Product of Sequence] 
Let $S = [s_1, s_2, \dots, s_r]$ be a sequence on $G$. Then, the $i$-th partial product $\pi_i(S)$ of $S$ is $s_1 s_2 \dots s_i$. We also set $\pi(S) \colon= \pi_r(S)$.
\end{definition}
\begin{example}[following the previous example]:
Let $G=\mathbb{Z}_{12}$ and $S = [s_1, s_2, \dots, s_r] = [1, 6, 9, 3, 1, 2, 5]$. Then $\pi_3(S) = 1 \cdot 6 \cdot 9 = 1+6+9 = 16 = 4$, $\pi(S) = 1 \cdot 6 \cdot 9 \cdot 3 \cdot 1 \cdot 2 \cdot 5 = 1+6+9+3+1+2+5 = 27 = 3$, and $r_S = r$.
\end{example}

\begin{definition}[Hamiltonian Sequence] 
We say that $S$ is \textit{hamiltonian} if $r_S = |G|$, $\pi(S) = \text{e}$, and the partial products $\pi_i(S)$, $i = 1, 2, \dots, r - 1$, are all distinct non-identity elements of $G$. 
\end{definition}
\begin{example}
Let $G=\mathbb{Z}_{12}$, $S = [1, 1, 1, 1, 1, 1, 1, 1, 1, 1, 1, 1]$. Then, $r_S = |G| = 12$, $\pi(S) = 12 = \text{e}$, and each of the partial products are all distinct non-identity elements of $G$.
\end{example}

\begin{definition}[M-Sequence] 
If $s_i \in M$, for $i = 1, 2, \dots, r$, then $S$ is called an $M$-sequence on $G$, and we define $\mathcal{H}(M, G)$ denote the set of all hamiltonian $M^*$-sequences on $G$. 
\end{definition}
\begin{example}[following the previous example]
Let $G=\mathbb{Z}_{12}$ and $M = \{0, 2, 3, 6\}$, then $S = [2, 3, 6, 3, 2, 2, 3]$ is a $M$-sequence on $G$.
\end{example}

\begin{definition}[Operations on Sequences] 
Let $S = [s_1, s_2, \dots, s_r]$, $T = [t_1, t_2, \dots, t_q]$ be sequences on $G$. Then:
\paragraph{(a)} 
$S^{-1} \colon= [s_r^{-1}, s_{r-1}^{-1}, \dots, s_1^{-1}]$, is called the \textit{inverse sequence} of \textit{sequence} $S$.
\paragraph{(b)} 
$l_S \colon= s_r$, $r\geq 2$.
\paragraph{(c)} 
We define $\overline{S}$ by $[s_1, s_2, \dots, s_{r-1}]$, $r\geq 2$, as the sequence $S$ without the last element.
\paragraph{(d)} 
We define $\hat{S}$ by

\hangindent=1.5em
$\hat{S} \colon= [s_2, s_3, \dots, s_{r-1}]$, $r\geq 3$.

\hangindent=1.5em
$\hat{S} \colon= \Box$, $r = 2$.
\paragraph{(e)} 
$ST \colon= [s_1, s_2, \dots, s_r, t_1, t_2, \dots, t_q]$
\paragraph{(f)} 
$(S, T)$

\hangindent=1.5em
$(S, T) \colon= [t_1]\hat{S}^{-1}[t_2]\hat{S} \dots [t_{q-2}]\hat{S}^{-1}[t_{q-1}]\hat{S}$ if $q \geq 3$ is odd.

\hangindent=1.5em
$(S, T) \colon=[t_1]\hat{S}^{-1}[t_2]\hat{S} \dots [t_{q-3}]\hat{S}^{-1}[t_{q-2}]\hat{S}$ if $q \geq 4$ is even.

\hangindent=1.5em
$(S, T) \colon=\Box$ if $q \in \{1, 2\}$.
\paragraph{(g)} 
$S^n$, for $n\in \mathbb{N} \cup \{0\}$

\hangindent=1.5em
$S^n \colon=\Box$ if $n = 0$.

\hangindent=1.5em
$S^n \colon= S^{n-1}S$ if $n \geq 1$.

\paragraph{(h)} 
$\Box S = S\Box = S$ and $\Box^n = \Box^m = \Box$.
\end{definition}

\begin{example}[follow the previous example]:\newline
Let $S = [2, 3, 6, 3, 2, 3]$, $T = [1, 2, 3]$, $R = [11, 10]$, and $U = [5, 6, 7, 8]$ be sequences on $G=\mathbb{Z}_{12}$.
Then:

$S^{-1} = [9, 10, 9, 6, 9, 10]$, 

$l_S = 3$, 

$\overline{S} = [2, 3, 6, 3, 2]$, 

\hangindent=1.5em
$\hat{S} = [3, 6, 3, 2]$.

\hangindent=1.5em
$\hat{R} = \Box$.

$ST = [2, 3, 6, 3, 2, 3, 1, 2, 3]$

\hangindent=1.5em
$(S, T) = [1, 10, 9, 6, 9, 2, 3, 6, 3, 2]$

\hangindent=1.5em
$(S, U) = [5, 10, 9, 6, 9, 6, 3, 6, 3, 2]$

\hangindent=1.5em
$(S, R) =\Box$.

\hangindent=1.5em
$S^{0} =\Box$.

\hangindent=1.5em
$S^{3} = [2, 3, 6, 3, 2, 3, 2, 3, 6, 3, 2, 3, 2, 3, 6, 3, 2, 3]$.
\end{example}

\begin{definition}[Cayley graph] 
Letting $M$ be a generating set of $G$, the \textit{Cayley graph} $\Gamma(G, M)$ is defined to be a graph such that $V(\Gamma(G, M)) = G$ and two vertices $x, y$ of $G$ are adjacent in $\Gamma(G, M)$ if and only if $xy^{-1} \in M^*$. 
\end{definition}
\begin{remark} 
$\Gamma(G, M)$ is a connected vertex symmetric graph, if and only if $M$ is a generating set for $G$. All Cayley graphs dealt with in this thesis will be assumed to have at least three vertices.
\end{remark}

\numberwithin{lemma}{section}
\begin{lemma}
For any surjective group homomorphism \( \varepsilon \colon G \to \overline{G} \), \( \varepsilon(x) = xH \)  \( \forall x \in G \), where \( \overline{G} \) = \( G/H \) as the quotient group. For any generating set \( X \) of \( G \), we have
\[
\overline{G} = \langle \varepsilon(X) \rangle.
\]
\end{lemma}
\begin{proof}
    Let \( G \) be a group and \( H \) a normal subgroup of \( G \), so the quotient group \( G / H \) is well-defined. Let \( X \) be a generating set of \( G \), meaning \( G = \langle X \rangle \). Define a group homomorphism \( \varepsilon: G \to G / H \) by \( \varepsilon(x) = xH \) for all \( x \in G \). We aim to prove that
    \[
    G / H = \langle \varepsilon(X) \rangle
    \]
    by double inclusion.
    By the definition of \( \varepsilon \), for any \( x \in X \), \( \varepsilon(x) = xH \in G / H \).  
    Thus, \( \varepsilon(X) \subseteq G / H \), and \( \langle \varepsilon(X) \rangle \) is a subgroup of \( G / H \).  
    Therefore, \( \langle \varepsilon(X) \rangle \subseteq G / H \).
    For any \( gH \in G / H \), there exists \( g \in G \).  
    Since \( X \) generates \( G \), \( g \) can be written as \( g = x_1^{k_1} x_2^{k_2} \dots x_n^{k_n} \) for some \( x_i \in X \) and \( k_i \in \mathbb{Z} \).  
    Applying the homomorphism \( \varepsilon \), we have:
    \[
    \varepsilon(g) = \varepsilon(x_1)^{k_1} \varepsilon(x_2)^{k_2} \dots \varepsilon(x_n)^{k_n},
    \]
    where \( \varepsilon(x_i) \in \varepsilon(X) \).  
    Hence, \( gH = \varepsilon(g) \in \langle \varepsilon(X) \rangle \). 
\end{proof}



\begin{lemma}
Let \( G \) be a finite abelian group, \( H \leq G \), \( g \in G \), \( g \notin H \). Let also \( j \in \mathbb{N} \) be the smallest positive integer such that \( g^j \in H \). Suppose that \( G = \langle H \cup \{g\} \rangle \). Then
\[
|H| \cdot j = |G|.
\]
\end{lemma}
\begin{proof}
Since \( G \) is abelian, \( H \trianglelefteq G \). Consider \( \overline{G} = G / H \) and let \( \varepsilon(x) = xH \), \( \forall x \in G \) be the natural homomorphism. Clearly, 
\[
\overline{G} = \langle \varepsilon(H \cup \{g\}) \rangle = \langle \{eH, gH\} \rangle = \langle gH \rangle = \langle \varepsilon(g) \rangle
\]
by Lemma 1.1. Further, \( |\varepsilon(g)| = j \). Indeed, as \( \varepsilon \) is a homomorphism, 
\[
(\varepsilon(g))^j = \varepsilon(g^j) = \varepsilon(e) = eH = H,
\]
and \( j \) is the smallest integer with this property as defined. Therefore, \( |\overline{G}| = j \), and \( |\overline{G}| \cdot |H| = |G| \) by Lagrange's theorem (\( |\overline{G}| = [G : H] \)), which implies
\[
|H| \cdot j = |G|.
\]
\end{proof}

\section{Theorems and Lemmas}

\numberwithin{lemma}{section}
\begin{lemma}
    The Cayley graph $F(G, M)$ is hamiltonian if and only if $\mathcal{H}(M, G)$ is not empty.
\end{lemma}

\begin{lemma}[Corrected version of Lemma 3.1 of~\cite{marusic1983}]
    Let $M$ be a generating set of an abelian group $G$ and $M'$ be a non-empty subset of $M_0$. If $S \in \mathcal{H}(M', \langle M' \rangle)$, then there exists a sequence $T$ on $G$ such that $\overline{S}T \in \mathcal{H}(M, G)$.
\end{lemma}
\begin{proof}
We proceed by induction on the cardinality of $M_0 \setminus M'$. The assertion of Lemma 3.1 is clearly true if $M_0 \setminus M' = \emptyset$. Let $M_0 \setminus M' \neq \emptyset$, $g \in M_0 \setminus M'$, $H = \langle M \setminus \{g\} \rangle$, and $j$ be the smallest positive integer such that $g^j \in H$. By the induction hypothesis, there exists a sequence $Q$ of elements of $H$ such that $\overline{S}Q, \in \mathcal{H}(M \setminus \{g\}, H)$. If $W = \overline{S}Q$, 
Let \( T \) be the sequence:
\[
T =
\begin{cases} 
\overline{Q}(W, [g]^j)[l_W][g^{-1}]^{j-1}, & \text{if \( j \) is odd;} \\
\overline{Q}(W, [g]^j)[g](\overline{W})^{-1}[g^{-1}]^{j-1}, & \text{if \( j \) is even.}
\end{cases}
\]

Then $\overline{S}T \in \mathcal{H}(M, G)$.

Now, we verify that \( \overline{S}T \) is a \textit{Hamiltonian} sequence on \( G \). Specifically, we must show:
\paragraph{(a)} 
$r_{\overline{S}T} = |G|$
\paragraph{(b)}
$\pi(\overline{S}T) = \text{e}$
\paragraph{(c)} 
The partial products $\pi_i(\overline{S}T) (i = 1, 2, \dots, r - 1)$ are all distinct non-identity elements of $G$.


\paragraph{(a)} As what we assumed for the inductive steps, $W = \overline{S}Q$ is already a \textit{hamiltonian} sequence on $H$, which means the length of $r_W = |H| = |G|/j$ by Lemma 1.2.
For the odd $j$ case:
\[
\overline{S}T = \overline{S} \mkern2mu \overline{Q}(W, [g]^j)[l_W][g^{-1}]^{j-1}
\]
\begin{align*}
r_{\overline{S}T} &= r_{\overline{S} \mkern2mu \overline{Q}} +r_{(W, [g]^j)} + r_{l_W} + r_{[g^{-1}]^{j-1}} \\
                  &= \frac{|G|}{j} - 1 + (j-1) \cdot \left(\frac{|G|}{j} - 2\right) + (j-1) + 1 + (j - 1) \\
                  &= |G|.
\end{align*}
For the even $j$ case:
\[
\overline{S}T = \overline{S}\mkern2mu\overline{Q}(W, [g]^j)[g](\overline{W})^{-1}[g^{-1}]^{j-1}
\]
\begin{align*}
r_{\overline{S}T} &= r_{\overline{S} \mkern2mu \overline{Q}} +r_{(W, [g]^j)} + r_{[g]} + r_{(\overline{W})^{-1}} + r_{[g^{-1}]^{j-1}} \\
                  &= \frac{|G|}{j} - 1 + (j-2)*(\frac{|G|}{j} - 2) + (j-2) + 1 + \frac{|G|}{j} - 1 + (j - 1)\\
                  &= |G|.
\end{align*}
Thus the length of sequence, $r_{\overline{S}T}$, in both cases regarding the parity of $j$, are identical with the of order of the group $G$.

\paragraph{(b)} We now verify that \( \pi(\overline{S}T) = \text{e} \), i.e., the product of all elements in the sequence \( \overline{S}T \) equals the identity element of \( G \). By the inductive hypothesis, \( W = \overline{S}Q \) is already a \textit{Hamiltonian} sequence on \( H \), which means
\[
\pi(\overline{S}Q) = \text{e}.
\]
For the odd $j$ case:
\[
\overline{S}T = \overline{S}\mkern2mu\overline{Q}(W, [g]^j)[l_W][g^{-1}]^{j-1}
\]
Given that $G$ is abelian, we can pair up $\overline{S}\mkern2mu\overline{Q}$ and $[l_W]$ to form $\pi(\overline{S}Q)= \text{e}$. Since $j$ is odd, there are $j-1$ times $[g]$ in sequence $(W, [g]^j)$, and we can pair up with $[g^{-1}]^{j-1}$ at the end, which means $\pi([g]^{j-1}[g^{-1}]^{j-1}) = \text{e}$. Also, since $j-1$ is even, as the alternating definition of $(W, [g]^j)$, $\widehat{\overline{S}Q}^{-1}$ and $\widehat{\overline{S}Q}$ each show up $\frac{j-1}{2}$ times, which means they can be canceled out. This is easy to observe as: (suppose $\widehat{\overline{S}Q} = [\alpha_1, \alpha_2, \dots, \alpha_n]$)
\begin{align*}
\widehat{\overline{S}Q}^{-1} \widehat{\overline{S}Q} &= [\alpha_n^{-1}, \alpha_{n-1}^{-1}, \dots, \alpha_1^{-1}] [\alpha_1, \alpha_2, \dots, \alpha_n] \\
                  &= \alpha_1 \alpha_1^{-1} \alpha_2 \alpha_2^{-1} \dots \alpha_n \alpha_n^{-1} \, (\text{by abelian}) \\
                  &= \text{e}.
\end{align*}
Thus, all the elements in $\overline{S}T$ canceled out into identity element, which means $\pi(\overline{S}T)= \text{e}$, and for the even $j$ case:
\[
\overline{S}T = \overline{S}\mkern2mu\overline{Q}(W, [g]^j)[g](\overline{W})^{-1}[g^{-1}]^{j-1}
\]
Since $j$ is even, $(W, [g]^j)$ has $j-2$ times $[g]$. Together with the following $[g]$, there are $j-1$ times $[g]$, which can be canceled out with the $[g^{-1}]^{j-1}$ sequence at the end of the formula. 
Additionally, with the same reason in the odd $j$ case, as the alternating definition of $(W, [g]^j)$, $\widehat{\overline{S}Q}^{-1}$ and $\widehat{\overline{S}Q}$ each show up $\frac{j-2}{2}$ times, which means they can also be canceled. 
Also,\[
\overline{W} = \overline{\overline{S}Q} = \overline{S}\mkern2mu\overline{Q} \implies (\overline{W})^{-1} = (\overline{S}\mkern2mu\overline{Q})^{-1}
\]
Hence the initial $\overline{S}\mkern2mu\overline{Q}$ can be canceled with $(\overline{W})^{-1}$. 
Thus, all the elements in $\overline{S}T$ canceled out into identity element, which means $\pi(\overline{S}T)= \text{e}$.

\paragraph{(c)} For the odd $j$ case:
\[
\overline{S}T = \overline{S}\mkern2mu\overline{Q}(W, [g]^j)[l_W][g^{-1}]^{j-1}
\]
To expand this expression, 
\[
\overline{S}T = \overline{S}\mkern2mu\overline{Q}[g]{\hat{W}}^{-1}[g]\hat{W}\dots[g]{\hat{W}}^{-1}[g]\hat{W}[l_W][g^{-1}]^{j-1}
\]
we can easily see that $\overline{S}\mkern2mu\overline{Q}$ is a sequence on $H$, and the first $[g]{\hat{W}}^{-1}$ is part of a sequence on $gH$, the second $[g]{\hat{W}}^{-1}$ is part of sequence on $g^2H$, and the last $[g]{\hat{W}}^{-1}$ combining $[l_W]$ is part of sequence on $g^{j-1}H$. The last $[g^{-1}]^{j-1}$ will trace the sequence through $g^{j-1}H$ to $gH$ and back to the end vertex to form a Hamiltonian circuit on $G$ by connecting the last identity element after the algorithm. Each partial fraction the partial product inside the sequence is product from different quotient groups, which means all are non-identity elements. 
For the even $j$ case:
\[
\overline{S}T = \overline{S}\mkern2mu\overline{Q}(W, [g]^j)[g](\overline{W})^{-1}[g^{-1}]^{j-1}
\]
After expanding this sequence, 
\[
\overline{S}T = \overline{S}\mkern2mu\overline{Q}[g]{\hat{W}}^{-1}[g]\hat{W}\dots[g]{\hat{W}}^{-1}[g]\hat{W}[g](\overline{W})^{-1}[g^{-1}]^{j-1}
\]
By a similar idea, we can easily see that $\overline{S}\mkern2mu\overline{Q}$ is a sequence on $H$, and the first $[g]{\hat{W}}^{-1}$ is part of sequence on $gH$, the second $[g]{\hat{W}}^{-1}$ is part of sequence on $g^2H$, and the last $[g]{\hat{W}}^{-1}$ is part of sequence on $g^{j-2}H$ by definition of even case. The following $[g](\overline{W})^{-1}$ part is in $g^{j-1}H$ to complete the traverse of elements in the last quotient group because of the definition of $(W, [g]^j)$ would end for $j-2$ if $j$ is even. At last, $[g^{-1}]^{j-1}$ will trace back the sequence through $g^{j-1}H$ to $gH$ and back to the end vertex to form a Hamiltonian circuit on $G$ by connecting the last identity element after the algorithm. Each partial fraction in the partial product inside the sequence is a product from different quotient groups, which means all are non-identity elements.
\end{proof}
To get a better understanding of this lemma, here is an example to get an idea how the algorithm works for a specific case.

\begin{example}
Let $G=\mathbb{Z}_{12}$, with generating set $M = \{0, 2, 3, 6\}$. Then:
\paragraph{} 
$M_0 = \{2, 3, 6\}$.
\paragraph{} 
Let $M' = \{2, 6\}$, and hence $M'^* = \{2, 10, 6\}$.
\paragraph{} 
$\langle M'\rangle = \{0, 2, 4, 6, 8, 10\}$
\paragraph{} 
For sequence from $\mathcal{H}(M', \langle M'\rangle)$:
$S = [2, 6, 2, 6, 2, 6]$, and $\overline{S} = [2, 6, 2, 6, 2]$.
\paragraph{} 
then there exists a sequence $T$ on $G$, s.t. 
$\overline{S}T \in \mathcal{H}(M, G)$.

\paragraph{} 
Next, to proceed the proof part, $g=3\in M_0 \setminus M'$, and $H = \langle M \setminus \{3\}\rangle = \langle\{0, 2, 6\}\rangle=\{0, 2, 4, 6, 8, 10\}$, and $j = 2$ s.t. $3^2 = 3+3 = 6\in H$. 

\paragraph{}
By the induction hypothesis there exists a sequence $Q$ on $H$,
s.t. $\overline{S}Q \in \mathcal{H}(M\setminus\{3\}, H)$

$\mathcal{H}(M\setminus\{3\}, H)$ denotes the set of all hamiltonian $(\{0, 2, 6\})^*$-sequences on $\{0, 2, 4, 6, 8, 10\}$, and $(\{0, 2, 6\})^* = \{2, 10, 6\}$. Let $Q=[6]$, and $W=\overline{S}Q = [2, 6, 2, 6, 2, 6]$.

Since $j=2$ is even, $
T =\overline{Q}(W, [g]^j)[g](\overline{W})^{-1}[g^{-1}]^{j-1}
$

$\overline{Q} = \Box$.

$(W, [g]^j) = \Box$, since $j=2$, following $\hat{W} = [6, 2, 6, 2]$, and $(\hat{W})^{-1} = [10, 6, 10, 6]$. 

$[g] = [3]$

$(\overline{W})^{-1} = [10, 6, 10, 6, 10]$

$[g^{-1}]^{j-1} = [3^{-1}]^{2-1} = [9]$

$T = \Box \Box [3][10, 6, 10, 6, 10][9] = [3, 10, 6, 10, 6, 10, 9]$.

$\overline{S}T = [2, 6, 2, 6, 2, 3, 10, 6, 10, 6, 10, 9]$, $r_{\overline{S}T} = 12 = |\mathbb{Z}_{12}|$, and $\pi(\overline{S}T) =2+6+2+6+2+3+10+6+10+6+10+9 = 72 = 12*6 = 0$. Additionally, each partial is $2, 8, 10, 4, 6, 9, 7, 1, 11, 5, 3, 12\neq 0$, matching the requirement that each partial product is distinct and not identity zero, except the last on, which $\pi(\overline{S}T)=0$.

\paragraph{}
Thus, $\overline{S}T$ is a \textit{hamiltonian} sequence.
\end{example}

\begin{theorem}
    Every connected Cayley graph of an abelian group of order at least three is \textit{hamiltonian}.
\end{theorem}

\begin{proof}
    By Lemma 3.1, it suffices to show that if $M$ is a generating set of an abelian group $G$ of order at least 3, then $\mathcal{H}(M, G) \neq \emptyset$. 
    \paragraph{}
    Case 1: If $M$ contains an element $x$ of order $n \geq 3$, then let $M' = \{x\}$ and $S = [x]^n$. 
    \paragraph{}
    Case 2:If $M$ contains no element of order at least 3, then it contains (since $G$ has order at least 3) two distinct elements $y, z$ of order 2. Then let $M' = \{y, z\}$ and $S = ([y][z])^2$. 
    \paragraph{}
    It follows by Lemma 3.2 that $\mathcal{H}(M, G) \neq \emptyset$.
\end{proof}

\begin{example}
    Let $G \cong V_4 \cong \mathbb{Z}_{2} \times \mathbb{Z}_{2}$. 
    \paragraph{}
    Let $M = \{(0, 1), (1, 0)\}$. In this case, we could focus on the non-trivial case 2 with element only order 2, and obviously $M$ generates $V_4$.
    \paragraph{}
    $M' = M = \{(0, 1), (1, 0)\}$, and 
    $S = ([(0, 1)],[(1, 0)])^2=[(0, 1),(1, 0),(0, 1),(1, 0)]$.
    \paragraph{}
    By Lemma 2, since $S\in \mathcal{H}(M', \langle M' \rangle) = 
    \mathcal{H}(M', V_4)$, there exists a sequence $T$ on $G$ such that $\overline{S}T\in\mathcal{H}(H, V_4)$.
\end{example}

\section*{Acknowledgements}
\begin{itemize}
  \item Thanks to my Honor Thesis advisor, Professor Denis Osin, for continuous support and guidance throughout this project.
  \item Previous foundational work by Dragan Maru\v{s}i\'c, which inspired and informed my research on Hamiltonian circuits in Cayley graphs.
\end{itemize}

\begin{thebibliography}{9}

\bibitem{marusic1983}
D.~Maru\v{s}i\'c, 
\textit{Hamiltonian Circuits in Cayley Graphs}, 
Discrete Mathematics, Volume 46, Issue 1, 1983, Pages 49–54.

\bibitem{stelow2017}
M.~Stelow, 
\textit{Hamiltonicity in Cayley Graphs and Digraphs of Finite Abelian Groups}, 
University of Chicago REU Paper, August 2017. Available at: \url{https://math.uchicago.edu/~may/REU2017/REUPapers/Stelow.pdf}

\bibitem{bajo2021}
E.~Bajo Calderon, 
\textit{An Exploration on the Hamiltonicity of Cayley Digraphs}, 
Master’s Thesis, Youngstown State University, 2021.\\
Available at: \url{https://etd.ohiolink.edu/acprod/odb_etd/ws/send_file/send?accession=ysu161982054497591&disposition=inline}
\end{thebibliography}
\end{document}













